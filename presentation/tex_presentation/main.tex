%----------------------------------------------------------------------------------------
%    PACKAGES AND THEMES
%----------------------------------------------------------------------------------------

\documentclass[aspectratio=169,xcolor=dvipsnames]{beamer}
\usepackage[T2A]{fontenc}
\usepackage[utf8]{inputenc}
\usepackage[russian]{babel}
\usetheme{SimplePlus}

\usepackage{hyperref}
\usepackage{graphicx} % Allows including images
\usepackage{booktabs} % Allows the use of \toprule, \midrule and \bottomrule in tables
\usepackage{amsmath} % Enhanced math support
\usepackage{amssymb} % Additional math symbols
\usepackage{mathtools} % Additional math tools

%----------------------------------------------------------------------------------------
%    TITLE PAGE
%----------------------------------------------------------------------------------------

\title{Аппроксимация поверхностных производных функций с применением интегральных операторов}

\author{Студенты: Иван Козырев, Лев Кулеваций \\
 Руководитель: Сетуха Алексей Викторович}

\institute
{
    Образовательный центр «Сириус» \\
    Проект по вычислительной математике
}
\date{\today} % Date, can be changed to a custom date

%----------------------------------------------------------------------------------------
%    PRESENTATION SLIDES
%----------------------------------------------------------------------------------------

\begin{document}

% Слайд 1: Титульный слайд
\begin{frame}
    \titlepage
\end{frame}

% Слайд 2: Постановка проблемы
\begin{frame}{Постановка проблемы}
    \begin{alertblock}{Цель исследования}
        \textbf{Асимптотическая проверка формулы погрешности} и \textbf{нахождение минимума} данной функции численно.
    \end{alertblock}
    
    \begin{block}{Параметры задачи}
        \begin{itemize}
            \item $h$ — фиксированный диаметр разбиения
            \item $\varepsilon$ — изменяемый параметр регуляризации
            \item Поиск оптимального $\varepsilon$ для минимизации погрешности
        \end{itemize}
    \end{block}
    
    \begin{figure}
        \centering
        \begin{minipage}{0.45\textwidth}
            \centering
            \textbf{[Заглушка: График функции 1 из ParaView]}
            \vspace{2cm}
        \end{minipage}
        \hfill
        \begin{minipage}{0.45\textwidth}
            \centering
            \textbf{[Заглушка: График функции 2 из ParaView]}
            \vspace{2cm}
        \end{minipage}
    \end{figure}
\end{frame}

% Слайд 3: Определение поверхностного градиента
\begin{frame}{Определение поверхностного градиента}
    \begin{block}{Поверхностный градиент}
        Для функции $f$, определённой на поверхности $S$:
        \[f(y) = f(x) + \langle A, y - x \rangle + o(|y - x|)\]
        где $A$ — поверхностный градиент в точке $x$.
    \end{block}
    
    \begin{block}{Интегральная формула вычисления}
        \[A = \nabla_S f(x) = \int_S K(x, y) f(y) \, dS(y)\]
        где $K(x, y)$ — ядро интегрального оператора.
    \end{block}
\end{frame}

% Слайд 4: Численная аппроксимация
\begin{frame}{Численная аппроксимация поверхностного градиента}
    \begin{block}{Формула численной аппроксимации}
        \[\tilde{A}_h = \sum_{j=1}^N K_\varepsilon(x, y_j) f(y_j) |S_j|\]
        где $y_j$ — центры треугольников, $|S_j|$ — их площади.
    \end{block}
    
    \begin{block}{Оценка погрешности}
        \begin{center}
            \textbf{[Заглушка для формулы погрешности из image copy 2.png]}
        \end{center}
    \end{block}
    
    \begin{block}{Диаметр разбиения}
        \[h = \max_{j=1,N} \sup_{x,y \in \sigma_j} |x - y|\]
    \end{block}
\end{frame}

% Слайд 5: Сравнение различных ядер
\begin{frame}{Сравнение различных ядер}
    \begin{figure}
        \centering
        \begin{minipage}{0.45\textwidth}
            \centering
            \textbf{[Заглушка: Сравнение 4 ядер для функции 1]}
            \vspace{3cm}
        \end{minipage}
        \hfill
        \begin{minipage}{0.45\textwidth}
            \centering
            \textbf{[Заглушка: Сравнение 4 ядер для функции 2]}
            \vspace{3cm}
        \end{minipage}
    \end{figure}
\end{frame}

% Слайд 6: Визуализация градиента на сфере
\begin{frame}{Визуализация поверхностного градиента}
    \begin{figure}
        \centering
        \begin{minipage}{0.45\textwidth}
            \centering
            \textbf{[Заглушка: Сфера с настоящим градиентом]}
            \vspace{4cm}
        \end{minipage}
        \hfill
        \begin{minipage}{0.45\textwidth}
            \centering
            \textbf{[Заглушка: Сфера с L2 ошибкой]}
            \vspace{4cm}
        \end{minipage}
    \end{figure}
\end{frame}

% Слайд 7: Дополнительная визуализация
\begin{frame}{Дополнительная визуализация}
    \begin{figure}
        \centering
        \begin{minipage}{0.45\textwidth}
            \centering
            \textbf{[Заглушка: Сфера с настоящим градиентом]}
            \vspace{4cm}
        \end{minipage}
        \hfill
        \begin{minipage}{0.45\textwidth}
            \centering
            \textbf{[Заглушка: Сфера с L2 ошибкой]}
            \vspace{4cm}
        \end{minipage}
    \end{figure}
\end{frame}

% Слайд 8: Теоретический вывод
\begin{frame}{Теоретический вывод зависимости $h$ от $\varepsilon$}
    \begin{block}{Исходная формула погрешности}
        \[E_f(\varepsilon, h) = a \cdot \varepsilon^p + b \cdot \frac{h^2}{\varepsilon^2}\]
    \end{block}
    
    \begin{alertblock}{Оптимальная зависимость}
        \[h_{opt} = \varepsilon^{\frac{2}{p+2}}\]
        где $p$ — степень в первом слагаемом формулы погрешности.
    \end{alertblock}
\end{frame}

% Слайд 9: Графики зависимости h от eps
\begin{frame}{Зависимость $h$ от $\varepsilon$}
    \begin{figure}
        \centering
        \begin{minipage}{0.45\textwidth}
            \centering
            \textbf{[Заглушка: Численно найденная]}
            \textbf{оптимальная зависимость для функции 1}
            \vspace{3cm}
        \end{minipage}
        \hfill
        \begin{minipage}{0.45\textwidth}
            \centering
            \textbf{[Заглушка: Численно найденная]}
            \textbf{оптимальная зависимость для функции 2}
            \vspace{3cm}
        \end{minipage}
    \end{figure}
\end{frame}

% Слайд 10: График L2 ошибки
\begin{frame}{Зависимость $L^2$ ошибки от $h$}
    \begin{figure}
        \centering
        \textbf{[Заглушка: График зависимости $L^2$ ошибки от $h$]}
        \vspace{5cm}
    \end{figure}
\end{frame}

% Слайд 11: Выводы
\begin{frame}{Выводы}
    \begin{block}{Основные результаты}
        \begin{itemize}
            \item Проведена асимптотическая проверка теоретической формулы погрешности
            \item Численно найдена оптимальная зависимость $h$ от $\varepsilon$
            \item Подтверждена теоретическая зависимость $h = \varepsilon^{2/(p+2)}$
            \item Исследовано поведение $L^2$ ошибки при различных параметрах дискретизации
        \end{itemize}
    \end{block}
    
    \begin{alertblock}{Практическое значение}
        Полученные результаты позволяют оптимально выбирать параметры численного метода для достижения минимальной погрешности аппроксимации поверхностных производных.
    \end{alertblock}
\end{frame}

% Слайд 12: Источники
\begin{frame}{Источники}
    \begin{thebibliography}{2}
        \bibitem{setuha}
        \textbf{Сетуха Алексей Викторович}
        \newblock Об аппроксимации поверхностных производных функций с применением интегральных операторов
        \newblock Научная статья
        
        \bibitem{eldredge}
        \textbf{Jeff D. Eldredge, Anthony Leonard, and Tim Colonius}
        \newblock A General Deterministic Treatment of Derivatives in Particle Methods
        \newblock Научная статья
    \end{thebibliography}
\end{frame>

% Слайд 13: Спасибо за внимание
\begin{frame}
    \Huge{\centerline{\textbf{Спасибо за внимание!}}}
\end{frame}

%----------------------------------------------------------------------------------------

\end{document}